% Options for packages loaded elsewhere
\PassOptionsToPackage{unicode}{hyperref}
\PassOptionsToPackage{hyphens}{url}
\documentclass[
  11pt,
]{article}
\usepackage{xcolor}
\usepackage[margin=1in]{geometry}
\usepackage{amsmath,amssymb}
\setcounter{secnumdepth}{-\maxdimen} % remove section numbering
\usepackage{iftex}
\ifPDFTeX
  \usepackage[T1]{fontenc}
  \usepackage[utf8]{inputenc}
  \usepackage{textcomp} % provide euro and other symbols
\else % if luatex or xetex
  \usepackage{unicode-math} % this also loads fontspec
  \defaultfontfeatures{Scale=MatchLowercase}
  \defaultfontfeatures[\rmfamily]{Ligatures=TeX,Scale=1}
\fi
\usepackage{lmodern}
\ifPDFTeX\else
  % xetex/luatex font selection
\fi
% Use upquote if available, for straight quotes in verbatim environments
\IfFileExists{upquote.sty}{\usepackage{upquote}}{}
\IfFileExists{microtype.sty}{% use microtype if available
  \usepackage[]{microtype}
  \UseMicrotypeSet[protrusion]{basicmath} % disable protrusion for tt fonts
}{}
\makeatletter
\@ifundefined{KOMAClassName}{% if non-KOMA class
  \IfFileExists{parskip.sty}{%
    \usepackage{parskip}
  }{% else
    \setlength{\parindent}{0pt}
    \setlength{\parskip}{6pt plus 2pt minus 1pt}}
}{% if KOMA class
  \KOMAoptions{parskip=half}}
\makeatother
\usepackage{color}
\usepackage{fancyvrb}
\newcommand{\VerbBar}{|}
\newcommand{\VERB}{\Verb[commandchars=\\\{\}]}
\DefineVerbatimEnvironment{Highlighting}{Verbatim}{commandchars=\\\{\}}
% Add ',fontsize=\small' for more characters per line
\usepackage{framed}
\definecolor{shadecolor}{RGB}{248,248,248}
\newenvironment{Shaded}{\begin{snugshade}}{\end{snugshade}}
\newcommand{\AlertTok}[1]{\textcolor[rgb]{0.94,0.16,0.16}{#1}}
\newcommand{\AnnotationTok}[1]{\textcolor[rgb]{0.56,0.35,0.01}{\textbf{\textit{#1}}}}
\newcommand{\AttributeTok}[1]{\textcolor[rgb]{0.13,0.29,0.53}{#1}}
\newcommand{\BaseNTok}[1]{\textcolor[rgb]{0.00,0.00,0.81}{#1}}
\newcommand{\BuiltInTok}[1]{#1}
\newcommand{\CharTok}[1]{\textcolor[rgb]{0.31,0.60,0.02}{#1}}
\newcommand{\CommentTok}[1]{\textcolor[rgb]{0.56,0.35,0.01}{\textit{#1}}}
\newcommand{\CommentVarTok}[1]{\textcolor[rgb]{0.56,0.35,0.01}{\textbf{\textit{#1}}}}
\newcommand{\ConstantTok}[1]{\textcolor[rgb]{0.56,0.35,0.01}{#1}}
\newcommand{\ControlFlowTok}[1]{\textcolor[rgb]{0.13,0.29,0.53}{\textbf{#1}}}
\newcommand{\DataTypeTok}[1]{\textcolor[rgb]{0.13,0.29,0.53}{#1}}
\newcommand{\DecValTok}[1]{\textcolor[rgb]{0.00,0.00,0.81}{#1}}
\newcommand{\DocumentationTok}[1]{\textcolor[rgb]{0.56,0.35,0.01}{\textbf{\textit{#1}}}}
\newcommand{\ErrorTok}[1]{\textcolor[rgb]{0.64,0.00,0.00}{\textbf{#1}}}
\newcommand{\ExtensionTok}[1]{#1}
\newcommand{\FloatTok}[1]{\textcolor[rgb]{0.00,0.00,0.81}{#1}}
\newcommand{\FunctionTok}[1]{\textcolor[rgb]{0.13,0.29,0.53}{\textbf{#1}}}
\newcommand{\ImportTok}[1]{#1}
\newcommand{\InformationTok}[1]{\textcolor[rgb]{0.56,0.35,0.01}{\textbf{\textit{#1}}}}
\newcommand{\KeywordTok}[1]{\textcolor[rgb]{0.13,0.29,0.53}{\textbf{#1}}}
\newcommand{\NormalTok}[1]{#1}
\newcommand{\OperatorTok}[1]{\textcolor[rgb]{0.81,0.36,0.00}{\textbf{#1}}}
\newcommand{\OtherTok}[1]{\textcolor[rgb]{0.56,0.35,0.01}{#1}}
\newcommand{\PreprocessorTok}[1]{\textcolor[rgb]{0.56,0.35,0.01}{\textit{#1}}}
\newcommand{\RegionMarkerTok}[1]{#1}
\newcommand{\SpecialCharTok}[1]{\textcolor[rgb]{0.81,0.36,0.00}{\textbf{#1}}}
\newcommand{\SpecialStringTok}[1]{\textcolor[rgb]{0.31,0.60,0.02}{#1}}
\newcommand{\StringTok}[1]{\textcolor[rgb]{0.31,0.60,0.02}{#1}}
\newcommand{\VariableTok}[1]{\textcolor[rgb]{0.00,0.00,0.00}{#1}}
\newcommand{\VerbatimStringTok}[1]{\textcolor[rgb]{0.31,0.60,0.02}{#1}}
\newcommand{\WarningTok}[1]{\textcolor[rgb]{0.56,0.35,0.01}{\textbf{\textit{#1}}}}
\usepackage{graphicx}
\makeatletter
\newsavebox\pandoc@box
\newcommand*\pandocbounded[1]{% scales image to fit in text height/width
  \sbox\pandoc@box{#1}%
  \Gscale@div\@tempa{\textheight}{\dimexpr\ht\pandoc@box+\dp\pandoc@box\relax}%
  \Gscale@div\@tempb{\linewidth}{\wd\pandoc@box}%
  \ifdim\@tempb\p@<\@tempa\p@\let\@tempa\@tempb\fi% select the smaller of both
  \ifdim\@tempa\p@<\p@\scalebox{\@tempa}{\usebox\pandoc@box}%
  \else\usebox{\pandoc@box}%
  \fi%
}
% Set default figure placement to htbp
\def\fps@figure{htbp}
\makeatother
\setlength{\emergencystretch}{3em} % prevent overfull lines
\providecommand{\tightlist}{%
  \setlength{\itemsep}{0pt}\setlength{\parskip}{0pt}}
\usepackage{bookmark}
\IfFileExists{xurl.sty}{\usepackage{xurl}}{} % add URL line breaks if available
\urlstyle{same}
\hypersetup{
  pdftitle={BIOL 683 --- Formative Assessment 1: Data Thinking + AI Collaboration},
  pdfauthor={Your Name (replace)},
  hidelinks,
  pdfcreator={LaTeX via pandoc}}

\title{BIOL 683 --- Formative Assessment 1: Data Thinking + AI
Collaboration}
\author{Your Name (replace)}
\date{2025-09-22}

\begin{document}
\maketitle

{
\setcounter{tocdepth}{2}
\tableofcontents
}
\begin{quote}
\textbf{Instructions (read first):} This is a \textbf{test‑style,
open‑notes, open‑AI} formative assessment. Knit to \textbf{PDF} for
submission. You may use AI tools (ChatGPT/Copilot, etc.) to assist with
code and writing, but \textbf{you are responsible for correctness}.
Wherever you used AI, \textbf{briefly note the prompt and what you
accepted/changed} in a footnote or an inline comment. Keep your writing
clear and concise.

\textbf{Policy reminders:} Show your reasoning, check assumptions,
visualize the data, and justify test choices. Avoid p‑hacking. Prefer
figures that \emph{show the data} and are accessible (e.g.,
color‑blind‑safe palettes).
\end{quote}

\subsection{Grading (20 points total)}\label{grading-20-points-total}

\begin{itemize}
\tightlist
\item
  \textbf{Completeness \& Reproducibility (4 pts):} The document knits;
  code is runnable and commented; seed set; session info included.
\item
  \textbf{Data Thinking (4 pts):} Clear description of data‑generating
  process and assumptions; appropriate diagnostics.
\item
  \textbf{Test Choice \& Justification (4 pts):} Correct tests selected;
  assumptions addressed; effect sizes and CIs reported.
\item
  \textbf{Figures \& Communication (4 pts):} Figures show the data,
  avoid chartjunk, and have informative captions.
\item
  \textbf{AI Use \& Reflection (4 pts):} Where AI helped/hurt; prompts
  summarized; independent judgment demonstrated.
\end{itemize}

\begin{center}\rule{0.5\linewidth}{0.5pt}\end{center}

\subsection{Section A --- Simulate or Load Data (4
pts)}\label{section-a-simulate-or-load-data-4-pts}

\textbf{Option 1 (simulate)}: Create two groups (n = 30 each) of a
biological measurement. - Group A: Normal with mean 10, SD 2. - Group B:
Right‑skewed (log‑normal) with \texttt{meanlog\ =\ 2.3},
\texttt{sdlog\ =\ 0.3}.

\textbf{Option 2 (load)}: Use a real dataset of your choice (link or
attach) and \textbf{briefly describe} the data‑generating process.

\begin{quote}
\textbf{Deliverables:} A short paragraph describing the data; a table
with sample size, mean, SD; and exploratory plots (histogram and
box/violin with points).
\end{quote}

\begin{Shaded}
\begin{Highlighting}[]
\CommentTok{\# }\AlertTok{TODO}\CommentTok{: Choose Option 1 (simulate) or Option 2 (load).}
\CommentTok{\# {-}{-}{-} Option 1: simulate example scaffold (edit as needed) {-}{-}{-}}
\NormalTok{n }\OtherTok{\textless{}{-}} \DecValTok{30}
\NormalTok{grpA }\OtherTok{\textless{}{-}} \FunctionTok{rnorm}\NormalTok{(n, }\AttributeTok{mean =} \DecValTok{10}\NormalTok{, }\AttributeTok{sd =} \DecValTok{2}\NormalTok{)}
\NormalTok{grpB }\OtherTok{\textless{}{-}} \FunctionTok{rlnorm}\NormalTok{(n, }\AttributeTok{meanlog =} \FloatTok{2.3}\NormalTok{, }\AttributeTok{sdlog =} \FloatTok{0.3}\NormalTok{)}
\NormalTok{group }\OtherTok{\textless{}{-}} \FunctionTok{factor}\NormalTok{(}\FunctionTok{rep}\NormalTok{(}\FunctionTok{c}\NormalTok{(}\StringTok{"A"}\NormalTok{,}\StringTok{"B"}\NormalTok{), }\AttributeTok{each =}\NormalTok{ n))}
\NormalTok{y }\OtherTok{\textless{}{-}} \FunctionTok{c}\NormalTok{(grpA, grpB)}
\NormalTok{df }\OtherTok{\textless{}{-}} \FunctionTok{data.frame}\NormalTok{(group, y)}
\FunctionTok{summary}\NormalTok{(df)}
\CommentTok{\# Plots (add labels/titles; consider viridisLite for colors)}
\FunctionTok{hist}\NormalTok{(grpA, }\AttributeTok{main =} \StringTok{"Group A histogram"}\NormalTok{, }\AttributeTok{xlab =} \StringTok{"Value"}\NormalTok{)}
\FunctionTok{hist}\NormalTok{(grpB, }\AttributeTok{main =} \StringTok{"Group B histogram"}\NormalTok{, }\AttributeTok{xlab =} \StringTok{"Value"}\NormalTok{)}
\FunctionTok{boxplot}\NormalTok{(y }\SpecialCharTok{\textasciitilde{}}\NormalTok{ group, }\AttributeTok{data =}\NormalTok{ df, }\AttributeTok{main =} \StringTok{"Group comparison"}\NormalTok{, }\AttributeTok{ylab =} \StringTok{"Value"}\NormalTok{)}
\end{Highlighting}
\end{Shaded}

\textbf{Briefly describe the data and the hypothesized biological
mechanism generating it.} \emph{(3--5 sentences.)}

\begin{center}\rule{0.5\linewidth}{0.5pt}\end{center}

\subsection{Section B --- Assumptions \& Diagnostics (4
pts)}\label{section-b-assumptions-diagnostics-4-pts}

\begin{enumerate}
\def\labelenumi{\arabic{enumi}.}
\tightlist
\item
  Use \textbf{histograms} and \textbf{QQ‑plots} to assess normality.
\item
  Use \textbf{Shapiro--Wilk} (with caution) and comment on sample‑size
  sensitivity.
\item
  If assumptions look shaky, propose a \textbf{transformation} (log,
  square‑root, arcsine for proportions) and \textbf{justify} it.
\end{enumerate}

\begin{Shaded}
\begin{Highlighting}[]
\CommentTok{\# }\AlertTok{TODO}\CommentTok{: Diagnostics and proposed transform}
\CommentTok{\# Example scaffolds (edit/extend)}
\FunctionTok{par}\NormalTok{(}\AttributeTok{mfrow =} \FunctionTok{c}\NormalTok{(}\DecValTok{1}\NormalTok{,}\DecValTok{2}\NormalTok{))}
\FunctionTok{qqnorm}\NormalTok{(grpA); }\FunctionTok{qqline}\NormalTok{(grpA)}
\FunctionTok{qqnorm}\NormalTok{(grpB); }\FunctionTok{qqline}\NormalTok{(grpB)}
\FunctionTok{shapiro.test}\NormalTok{(grpA)}
\FunctionTok{shapiro.test}\NormalTok{(grpB)}
\CommentTok{\# Example transform:}
\NormalTok{y\_log }\OtherTok{\textless{}{-}} \ControlFlowTok{if}\NormalTok{ (}\FunctionTok{all}\NormalTok{(y }\SpecialCharTok{\textgreater{}} \DecValTok{0}\NormalTok{)) }\FunctionTok{log}\NormalTok{(y) }\ControlFlowTok{else}\NormalTok{ y  }\CommentTok{\# log only if positive}
\end{Highlighting}
\end{Shaded}

\textbf{Write 4--6 sentences} interpreting your diagnostics and whether
a transform is warranted.

\begin{center}\rule{0.5\linewidth}{0.5pt}\end{center}

\subsection{Section C --- Statistical Testing \& Estimation (8
pts)}\label{section-c-statistical-testing-estimation-8-pts}

\textbf{Goal:} Test whether groups differ in central tendency and
communicate uncertainty.

\begin{enumerate}
\def\labelenumi{\arabic{enumi}.}
\tightlist
\item
  Compare A vs B using \textbf{Welch's two‑sample t‑test} on raw and (if
  appropriate) transformed data. Report \textbf{effect size} (e.g., mean
  diff and Hedges' g) and \textbf{95\% CI}.
\item
  Also test with \textbf{Mann--Whitney (Wilcoxon rank‑sum)} and discuss
  agreement/disagreement.
\item
  Plot \textbf{group means with 95\% CIs} and \textbf{a data‑showing
  figure} (e.g., box/violin with jitter). Use an \textbf{accessible
  palette} (e.g., \texttt{viridisLite}).
\end{enumerate}

\begin{Shaded}
\begin{Highlighting}[]
\CommentTok{\# }\AlertTok{TODO}\CommentTok{: Analyses}
\CommentTok{\# Welch t{-}test}
\FunctionTok{t.test}\NormalTok{(y }\SpecialCharTok{\textasciitilde{}}\NormalTok{ group, }\AttributeTok{data =}\NormalTok{ df)}
\CommentTok{\# If transformed:}
\FunctionTok{t.test}\NormalTok{(y\_log }\SpecialCharTok{\textasciitilde{}}\NormalTok{ group, }\AttributeTok{data =} \FunctionTok{transform}\NormalTok{(df, }\AttributeTok{y\_log =} \FunctionTok{ifelse}\NormalTok{(y}\SpecialCharTok{\textgreater{}}\DecValTok{0}\NormalTok{, }\FunctionTok{log}\NormalTok{(y), }\ConstantTok{NA}\NormalTok{)), }\AttributeTok{na.action =}\NormalTok{ na.omit)}

\CommentTok{\# Mann–Whitney}
\FunctionTok{wilcox.test}\NormalTok{(y }\SpecialCharTok{\textasciitilde{}}\NormalTok{ group, }\AttributeTok{data =}\NormalTok{ df, }\AttributeTok{exact =} \ConstantTok{FALSE}\NormalTok{)}

\CommentTok{\# Effect size (Hedges\textquotesingle{} g) — quick implementation}
\NormalTok{hedges\_g }\OtherTok{\textless{}{-}} \ControlFlowTok{function}\NormalTok{(x, y)\{}
\NormalTok{  nx }\OtherTok{\textless{}{-}} \FunctionTok{length}\NormalTok{(x); ny }\OtherTok{\textless{}{-}} \FunctionTok{length}\NormalTok{(y)}
\NormalTok{  sx2 }\OtherTok{\textless{}{-}} \FunctionTok{var}\NormalTok{(x); sy2 }\OtherTok{\textless{}{-}} \FunctionTok{var}\NormalTok{(y)}
\NormalTok{  sp }\OtherTok{\textless{}{-}} \FunctionTok{sqrt}\NormalTok{(((nx}\DecValTok{{-}1}\NormalTok{)}\SpecialCharTok{*}\NormalTok{sx2 }\SpecialCharTok{+}\NormalTok{ (ny}\DecValTok{{-}1}\NormalTok{)}\SpecialCharTok{*}\NormalTok{sy2)}\SpecialCharTok{/}\NormalTok{(nx}\SpecialCharTok{+}\NormalTok{ny}\DecValTok{{-}2}\NormalTok{))}
\NormalTok{  g }\OtherTok{\textless{}{-}}\NormalTok{ (}\FunctionTok{mean}\NormalTok{(x) }\SpecialCharTok{{-}} \FunctionTok{mean}\NormalTok{(y))}\SpecialCharTok{/}\NormalTok{sp}
\NormalTok{  J }\OtherTok{\textless{}{-}} \DecValTok{1} \SpecialCharTok{{-}} \DecValTok{3}\SpecialCharTok{/}\NormalTok{(}\DecValTok{4}\SpecialCharTok{*}\NormalTok{(nx}\SpecialCharTok{+}\NormalTok{ny)}\SpecialCharTok{{-}}\DecValTok{9}\NormalTok{)  }\CommentTok{\# small{-}sample correction}
\NormalTok{  g }\SpecialCharTok{*}\NormalTok{ J}
\NormalTok{\}}
\FunctionTok{with}\NormalTok{(df, }\FunctionTok{hedges\_g}\NormalTok{(y[group}\SpecialCharTok{==}\StringTok{"A"}\NormalTok{], y[group}\SpecialCharTok{==}\StringTok{"B"}\NormalTok{]))}

\CommentTok{\# CIs plot (means +/{-} 1.96*SE)}
\NormalTok{agg }\OtherTok{\textless{}{-}} \FunctionTok{aggregate}\NormalTok{(y }\SpecialCharTok{\textasciitilde{}}\NormalTok{ group, df, }\ControlFlowTok{function}\NormalTok{(v) }\FunctionTok{c}\NormalTok{(}\AttributeTok{mean=}\FunctionTok{mean}\NormalTok{(v), }\AttributeTok{se=}\FunctionTok{sd}\NormalTok{(v)}\SpecialCharTok{/}\FunctionTok{sqrt}\NormalTok{(}\FunctionTok{length}\NormalTok{(v))))}
\NormalTok{agg }\OtherTok{\textless{}{-}} \FunctionTok{data.frame}\NormalTok{(}\AttributeTok{group =}\NormalTok{ agg}\SpecialCharTok{$}\NormalTok{group, }\AttributeTok{mean =}\NormalTok{ agg}\SpecialCharTok{$}\NormalTok{y[, }\StringTok{"mean"}\NormalTok{], }\AttributeTok{se =}\NormalTok{ agg}\SpecialCharTok{$}\NormalTok{y[, }\StringTok{"se"}\NormalTok{])}
\NormalTok{agg}\SpecialCharTok{$}\NormalTok{lower }\OtherTok{\textless{}{-}}\NormalTok{ agg}\SpecialCharTok{$}\NormalTok{mean }\SpecialCharTok{{-}} \FloatTok{1.96}\SpecialCharTok{*}\NormalTok{agg}\SpecialCharTok{$}\NormalTok{se}
\NormalTok{agg}\SpecialCharTok{$}\NormalTok{upper }\OtherTok{\textless{}{-}}\NormalTok{ agg}\SpecialCharTok{$}\NormalTok{mean }\SpecialCharTok{+} \FloatTok{1.96}\SpecialCharTok{*}\NormalTok{agg}\SpecialCharTok{$}\NormalTok{se}
\FunctionTok{print}\NormalTok{(agg)}

\CommentTok{\# Simple CI plot (base R)}
\FunctionTok{plot}\NormalTok{(agg}\SpecialCharTok{$}\NormalTok{group, agg}\SpecialCharTok{$}\NormalTok{mean, }\AttributeTok{ylim =} \FunctionTok{range}\NormalTok{(}\FunctionTok{c}\NormalTok{(agg}\SpecialCharTok{$}\NormalTok{lower, agg}\SpecialCharTok{$}\NormalTok{upper)), }\AttributeTok{xlab =} \StringTok{"Group"}\NormalTok{, }\AttributeTok{ylab =} \StringTok{"Mean with 95\% CI"}\NormalTok{, }\AttributeTok{pch =} \DecValTok{19}\NormalTok{)}
\FunctionTok{arrows}\NormalTok{(}\AttributeTok{x0 =} \DecValTok{1}\SpecialCharTok{:}\DecValTok{2}\NormalTok{, }\AttributeTok{y0 =}\NormalTok{ agg}\SpecialCharTok{$}\NormalTok{lower, }\AttributeTok{x1 =} \DecValTok{1}\SpecialCharTok{:}\DecValTok{2}\NormalTok{, }\AttributeTok{y1 =}\NormalTok{ agg}\SpecialCharTok{$}\NormalTok{upper, }\AttributeTok{angle =} \DecValTok{90}\NormalTok{, }\AttributeTok{code =} \DecValTok{3}\NormalTok{, }\AttributeTok{length =} \FloatTok{0.05}\NormalTok{)}
\end{Highlighting}
\end{Shaded}

\textbf{Interpretation (≤150 words):} Are results consistent across
tests and (if used) transformations? What do the \textbf{effect sizes}
and \textbf{CIs} suggest about biological relevance?

\begin{center}\rule{0.5\linewidth}{0.5pt}\end{center}

\subsection{Section D --- Figure Design \& Accessibility (2
pts)}\label{section-d-figure-design-accessibility-2-pts}

Produce \textbf{one publication‑quality figure} comparing groups that
\textbf{shows the data} (e.g., box/violin + jitter). Use a
color‑blind‑safe palette (e.g., \texttt{viridisLite::viridis(n=2)}). Add
an informative caption stating the key take‑home message.

\begin{Shaded}
\begin{Highlighting}[]
\CommentTok{\# }\AlertTok{TODO}\CommentTok{: Publishable figure (base or ggplot2 ok). Example (base):}
\NormalTok{cols }\OtherTok{\textless{}{-}}\NormalTok{ viridisLite}\SpecialCharTok{::}\FunctionTok{viridis}\NormalTok{(}\DecValTok{2}\NormalTok{)}
\FunctionTok{stripchart}\NormalTok{(y }\SpecialCharTok{\textasciitilde{}}\NormalTok{ group, }\AttributeTok{data =}\NormalTok{ df, }\AttributeTok{vertical =} \ConstantTok{TRUE}\NormalTok{, }\AttributeTok{pch =} \DecValTok{16}\NormalTok{, }\AttributeTok{col =} \FunctionTok{adjustcolor}\NormalTok{(cols[group], }\FloatTok{0.7}\NormalTok{), }\AttributeTok{method =} \StringTok{"jitter"}\NormalTok{, }\AttributeTok{main =} \StringTok{"Group comparison (data shown)"}\NormalTok{, }\AttributeTok{ylab =} \StringTok{"Value"}\NormalTok{)}
\FunctionTok{boxplot}\NormalTok{(y }\SpecialCharTok{\textasciitilde{}}\NormalTok{ group, }\AttributeTok{data =}\NormalTok{ df, }\AttributeTok{add =} \ConstantTok{TRUE}\NormalTok{, }\AttributeTok{border =} \StringTok{"gray30"}\NormalTok{, }\AttributeTok{col =} \ConstantTok{NA}\NormalTok{)}
\end{Highlighting}
\end{Shaded}

\textbf{Caption (2--3 sentences)} explaining the figure and what readers
should notice.

\begin{center}\rule{0.5\linewidth}{0.5pt}\end{center}

\subsection{Section E --- AI Use \& Reflection (2
pts)}\label{section-e-ai-use-reflection-2-pts}

\textbf{Briefly document} (bullets are fine): - Which prompts you used
(paste 1--2 best prompts). - Where AI helped you move faster. - One
place AI was wrong, unclear, or needed correction---and how you fixed
it.

\begin{center}\rule{0.5\linewidth}{0.5pt}\end{center}

\subsection{Reproducibility Appendix}\label{reproducibility-appendix}

\begin{Shaded}
\begin{Highlighting}[]
\FunctionTok{sessionInfo}\NormalTok{()}
\end{Highlighting}
\end{Shaded}

\begin{verbatim}
## R version 4.5.1 (2025-06-13)
## Platform: aarch64-apple-darwin20
## Running under: macOS Sequoia 15.5
## 
## Matrix products: default
## BLAS:   /Library/Frameworks/R.framework/Versions/4.5-arm64/Resources/lib/libRblas.0.dylib 
## LAPACK: /Library/Frameworks/R.framework/Versions/4.5-arm64/Resources/lib/libRlapack.dylib;  LAPACK version 3.12.1
## 
## locale:
## [1] en_US.UTF-8/en_US.UTF-8/en_US.UTF-8/C/en_US.UTF-8/en_US.UTF-8
## 
## time zone: America/Chicago
## tzcode source: internal
## 
## attached base packages:
## [1] stats     graphics  grDevices utils     datasets  methods   base     
## 
## loaded via a namespace (and not attached):
##  [1] compiler_4.5.1    fastmap_1.2.0     cli_3.6.5         tools_4.5.1      
##  [5] htmltools_0.5.8.1 rstudioapi_0.17.1 yaml_2.3.10       rmarkdown_2.29   
##  [9] knitr_1.50        xfun_0.52         digest_0.6.37     rlang_1.1.6      
## [13] evaluate_1.0.4
\end{verbatim}

\subsubsection{Notes on Good Practice (skim before you
start)}\label{notes-on-good-practice-skim-before-you-start}

\begin{itemize}
\tightlist
\item
  Prefer tests aligned to your \textbf{data‑generating process} and
  \textbf{assumptions}.
\item
  If you transform, say why---and consider confirming with a rank‑based
  test.
\item
  Figures should \textbf{show the data}, avoid distortion, and remain
  legible in grayscale.
\item
  When in doubt, simulate. Use AI to write scaffolding code, but
  \textbf{inspect outputs} carefully.
\end{itemize}

\end{document}
