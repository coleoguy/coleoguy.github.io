\PassOptionsToPackage{unicode=true}{hyperref} % options for packages loaded elsewhere
\PassOptionsToPackage{hyphens}{url}
%
\documentclass[]{article}
\usepackage{lmodern}
\usepackage{amssymb,amsmath}
\usepackage{ifxetex,ifluatex}
\usepackage{fixltx2e} % provides \textsubscript
\ifnum 0\ifxetex 1\fi\ifluatex 1\fi=0 % if pdftex
  \usepackage[T1]{fontenc}
  \usepackage[utf8]{inputenc}
  \usepackage{textcomp} % provides euro and other symbols
\else % if luatex or xelatex
  \usepackage{unicode-math}
  \defaultfontfeatures{Ligatures=TeX,Scale=MatchLowercase}
\fi
% use upquote if available, for straight quotes in verbatim environments
\IfFileExists{upquote.sty}{\usepackage{upquote}}{}
% use microtype if available
\IfFileExists{microtype.sty}{%
\usepackage[]{microtype}
\UseMicrotypeSet[protrusion]{basicmath} % disable protrusion for tt fonts
}{}
\IfFileExists{parskip.sty}{%
\usepackage{parskip}
}{% else
\setlength{\parindent}{0pt}
\setlength{\parskip}{6pt plus 2pt minus 1pt}
}
\usepackage{hyperref}
\hypersetup{
            pdftitle={PCA Tutorial},
            pdfauthor={Heath Blackmon},
            pdfborder={0 0 0},
            breaklinks=true}
\urlstyle{same}  % don't use monospace font for urls
\usepackage[margin=1in]{geometry}
\usepackage{color}
\usepackage{fancyvrb}
\newcommand{\VerbBar}{|}
\newcommand{\VERB}{\Verb[commandchars=\\\{\}]}
\DefineVerbatimEnvironment{Highlighting}{Verbatim}{commandchars=\\\{\}}
% Add ',fontsize=\small' for more characters per line
\usepackage{framed}
\definecolor{shadecolor}{RGB}{248,248,248}
\newenvironment{Shaded}{\begin{snugshade}}{\end{snugshade}}
\newcommand{\AlertTok}[1]{\textcolor[rgb]{0.94,0.16,0.16}{#1}}
\newcommand{\AnnotationTok}[1]{\textcolor[rgb]{0.56,0.35,0.01}{\textbf{\textit{#1}}}}
\newcommand{\AttributeTok}[1]{\textcolor[rgb]{0.77,0.63,0.00}{#1}}
\newcommand{\BaseNTok}[1]{\textcolor[rgb]{0.00,0.00,0.81}{#1}}
\newcommand{\BuiltInTok}[1]{#1}
\newcommand{\CharTok}[1]{\textcolor[rgb]{0.31,0.60,0.02}{#1}}
\newcommand{\CommentTok}[1]{\textcolor[rgb]{0.56,0.35,0.01}{\textit{#1}}}
\newcommand{\CommentVarTok}[1]{\textcolor[rgb]{0.56,0.35,0.01}{\textbf{\textit{#1}}}}
\newcommand{\ConstantTok}[1]{\textcolor[rgb]{0.00,0.00,0.00}{#1}}
\newcommand{\ControlFlowTok}[1]{\textcolor[rgb]{0.13,0.29,0.53}{\textbf{#1}}}
\newcommand{\DataTypeTok}[1]{\textcolor[rgb]{0.13,0.29,0.53}{#1}}
\newcommand{\DecValTok}[1]{\textcolor[rgb]{0.00,0.00,0.81}{#1}}
\newcommand{\DocumentationTok}[1]{\textcolor[rgb]{0.56,0.35,0.01}{\textbf{\textit{#1}}}}
\newcommand{\ErrorTok}[1]{\textcolor[rgb]{0.64,0.00,0.00}{\textbf{#1}}}
\newcommand{\ExtensionTok}[1]{#1}
\newcommand{\FloatTok}[1]{\textcolor[rgb]{0.00,0.00,0.81}{#1}}
\newcommand{\FunctionTok}[1]{\textcolor[rgb]{0.00,0.00,0.00}{#1}}
\newcommand{\ImportTok}[1]{#1}
\newcommand{\InformationTok}[1]{\textcolor[rgb]{0.56,0.35,0.01}{\textbf{\textit{#1}}}}
\newcommand{\KeywordTok}[1]{\textcolor[rgb]{0.13,0.29,0.53}{\textbf{#1}}}
\newcommand{\NormalTok}[1]{#1}
\newcommand{\OperatorTok}[1]{\textcolor[rgb]{0.81,0.36,0.00}{\textbf{#1}}}
\newcommand{\OtherTok}[1]{\textcolor[rgb]{0.56,0.35,0.01}{#1}}
\newcommand{\PreprocessorTok}[1]{\textcolor[rgb]{0.56,0.35,0.01}{\textit{#1}}}
\newcommand{\RegionMarkerTok}[1]{#1}
\newcommand{\SpecialCharTok}[1]{\textcolor[rgb]{0.00,0.00,0.00}{#1}}
\newcommand{\SpecialStringTok}[1]{\textcolor[rgb]{0.31,0.60,0.02}{#1}}
\newcommand{\StringTok}[1]{\textcolor[rgb]{0.31,0.60,0.02}{#1}}
\newcommand{\VariableTok}[1]{\textcolor[rgb]{0.00,0.00,0.00}{#1}}
\newcommand{\VerbatimStringTok}[1]{\textcolor[rgb]{0.31,0.60,0.02}{#1}}
\newcommand{\WarningTok}[1]{\textcolor[rgb]{0.56,0.35,0.01}{\textbf{\textit{#1}}}}
\usepackage{graphicx,grffile}
\makeatletter
\def\maxwidth{\ifdim\Gin@nat@width>\linewidth\linewidth\else\Gin@nat@width\fi}
\def\maxheight{\ifdim\Gin@nat@height>\textheight\textheight\else\Gin@nat@height\fi}
\makeatother
% Scale images if necessary, so that they will not overflow the page
% margins by default, and it is still possible to overwrite the defaults
% using explicit options in \includegraphics[width, height, ...]{}
\setkeys{Gin}{width=\maxwidth,height=\maxheight,keepaspectratio}
\setlength{\emergencystretch}{3em}  % prevent overfull lines
\providecommand{\tightlist}{%
  \setlength{\itemsep}{0pt}\setlength{\parskip}{0pt}}
\setcounter{secnumdepth}{0}
% Redefines (sub)paragraphs to behave more like sections
\ifx\paragraph\undefined\else
\let\oldparagraph\paragraph
\renewcommand{\paragraph}[1]{\oldparagraph{#1}\mbox{}}
\fi
\ifx\subparagraph\undefined\else
\let\oldsubparagraph\subparagraph
\renewcommand{\subparagraph}[1]{\oldsubparagraph{#1}\mbox{}}
\fi

% set default figure placement to htbp
\makeatletter
\def\fps@figure{htbp}
\makeatother


\title{PCA Tutorial}
\author{Heath Blackmon}
\date{4/12/2018}

\begin{document}
\maketitle

{
\setcounter{tocdepth}{3}
\tableofcontents
}
\hypertarget{packages-and-data}{%
\subsubsection{Packages and data}\label{packages-and-data}}

Today we will use two new packages

\begin{Shaded}
\begin{Highlighting}[]
\KeywordTok{install.packages}\NormalTok{(}\StringTok{"car"}\NormalTok{)}
\KeywordTok{install.packages}\NormalTok{(}\StringTok{"FactoMineR"}\NormalTok{)}
\end{Highlighting}
\end{Shaded}

\begin{enumerate}
\def\labelenumi{\arabic{enumi})}
\tightlist
\item
  Load the iris data and take a look at what you have
\end{enumerate}

\begin{Shaded}
\begin{Highlighting}[]
\KeywordTok{data}\NormalTok{(}\StringTok{'iris'}\NormalTok{)}
\KeywordTok{head}\NormalTok{(iris)}
\end{Highlighting}
\end{Shaded}

\begin{verbatim}
##   Sepal.Length Sepal.Width Petal.Length Petal.Width Species
## 1          5.1         3.5          1.4         0.2  setosa
## 2          4.9         3.0          1.4         0.2  setosa
## 3          4.7         3.2          1.3         0.2  setosa
## 4          4.6         3.1          1.5         0.2  setosa
## 5          5.0         3.6          1.4         0.2  setosa
## 6          5.4         3.9          1.7         0.4  setosa
\end{verbatim}

\hypertarget{basic-pca}{%
\subsubsection{Basic PCA}\label{basic-pca}}

\begin{enumerate}
\def\labelenumi{\arabic{enumi})}
\setcounter{enumi}{1}
\tightlist
\item
  Perform the PCA using the base function \texttt{prcomp} use the
  assignment operator \texttt{\textless{}-} to send this to a new
  variable named \texttt{pca}.
\end{enumerate}

\begin{Shaded}
\begin{Highlighting}[]
\NormalTok{pca <-}\StringTok{ }\KeywordTok{prcomp}\NormalTok{(iris[, }\DecValTok{1}\OperatorTok{:}\DecValTok{4}\NormalTok{])}
\end{Highlighting}
\end{Shaded}

\begin{enumerate}
\def\labelenumi{\arabic{enumi})}
\setcounter{enumi}{2}
\tightlist
\item
  Review the object that is returned. You can see the names for the
  different elements in the list using the \texttt{names} function
  (\texttt{names(pca)}).
\end{enumerate}

\begin{Shaded}
\begin{Highlighting}[]
\KeywordTok{names}\NormalTok{(pca)}
\end{Highlighting}
\end{Shaded}

\begin{verbatim}
## [1] "sdev"     "rotation" "center"   "scale"    "x"
\end{verbatim}

\begin{enumerate}
\def\labelenumi{\arabic{enumi})}
\setcounter{enumi}{3}
\tightlist
\item
  A scree plot shows us the proportion of variance explained by each
  principal component. Use the first element \texttt{pca\$sdev} along
  with the \texttt{sum} function and square function (\texttt{\^{}2}) to
  create a scree plot.
\end{enumerate}

\begin{Shaded}
\begin{Highlighting}[]
\NormalTok{y <-}\StringTok{ }\NormalTok{pca}\OperatorTok{$}\NormalTok{sdev}\OperatorTok{^}\DecValTok{2} \OperatorTok{/}\StringTok{ }\KeywordTok{sum}\NormalTok{(pca}\OperatorTok{$}\NormalTok{sdev}\OperatorTok{^}\DecValTok{2}\NormalTok{) }\OperatorTok{*}\StringTok{ }\DecValTok{100}
\KeywordTok{barplot}\NormalTok{(y, }
        \DataTypeTok{names.arg=}\KeywordTok{c}\NormalTok{(}\StringTok{"PC1"}\NormalTok{, }\StringTok{"PC2"}\NormalTok{, }\StringTok{"PC3"}\NormalTok{, }\StringTok{"PC4"}\NormalTok{),}
        \DataTypeTok{ylab =} \StringTok{"Percent Variance Captured"}\NormalTok{,}
        \DataTypeTok{xlab =} \StringTok{""}\NormalTok{,}
        \DataTypeTok{main =} \StringTok{""}\NormalTok{)}
\end{Highlighting}
\end{Shaded}

\includegraphics{pca-complete_files/figure-latex/unnamed-chunk-5-1.pdf}

\hypertarget{results-of-pca}{%
\subsubsection{Results of PCA}\label{results-of-pca}}

\begin{enumerate}
\def\labelenumi{\arabic{enumi})}
\setcounter{enumi}{4}
\tightlist
\item
  Use the fifth element (\texttt{pca\$x}) to create a plot of the
  samples in principal component space. Color each point based on its
  species identity.
\end{enumerate}

-hint: factors have values of 1, 2, 3. So what would you get if you ran
this code

\begin{Shaded}
\begin{Highlighting}[]
\KeywordTok{plot}\NormalTok{(}\DataTypeTok{x =}\NormalTok{ pca}\OperatorTok{$}\NormalTok{x[, }\DecValTok{1}\NormalTok{],}
     \DataTypeTok{y =}\NormalTok{ pca}\OperatorTok{$}\NormalTok{x[, }\DecValTok{2}\NormalTok{],}
     \DataTypeTok{col =} \KeywordTok{rainbow}\NormalTok{(}\DecValTok{3}\NormalTok{)[iris}\OperatorTok{$}\NormalTok{Species],}
     \DataTypeTok{xlab =} \StringTok{"PC1"}\NormalTok{,}
     \DataTypeTok{ylab =} \StringTok{"PC2"}\NormalTok{,}
     \DataTypeTok{main =} \StringTok{"PCA of iris data"}\NormalTok{)}
\end{Highlighting}
\end{Shaded}

\includegraphics{pca-complete_files/figure-latex/unnamed-chunk-6-1.pdf}

\hypertarget{loadings-and-variables-factor-map}{%
\subsubsection{Loadings and variables factor
map}\label{loadings-and-variables-factor-map}}

The second element (\texttt{pca\$rotation}) has loadings -- this is the
correlation between each of your raw measures and the sample scores for
each principal component. Sometimes people will report these raw
correaltions.

\begin{Shaded}
\begin{Highlighting}[]
\NormalTok{pca}\OperatorTok{$}\NormalTok{rotation}
\end{Highlighting}
\end{Shaded}

\begin{verbatim}
##                      PC1         PC2         PC3        PC4
## Sepal.Length  0.36138659 -0.65658877  0.58202985  0.3154872
## Sepal.Width  -0.08452251 -0.73016143 -0.59791083 -0.3197231
## Petal.Length  0.85667061  0.17337266 -0.07623608 -0.4798390
## Petal.Width   0.35828920  0.07548102 -0.54583143  0.7536574
\end{verbatim}

However, I think a more intuitive measure is how much of the variance in
the raw measure is captured by a principal component to get this simply
square these values.

\begin{Shaded}
\begin{Highlighting}[]
\NormalTok{pca}\OperatorTok{$}\NormalTok{rotation}\OperatorTok{^}\DecValTok{2}
\end{Highlighting}
\end{Shaded}

\begin{verbatim}
##                      PC1         PC2         PC3        PC4
## Sepal.Length 0.130600269 0.431108815 0.338758748 0.09953217
## Sepal.Width  0.007144055 0.533135721 0.357497361 0.10222286
## Petal.Length 0.733884527 0.030058080 0.005811939 0.23024545
## Petal.Width  0.128371149 0.005697384 0.297931952 0.56799951
\end{verbatim}

\begin{enumerate}
\def\labelenumi{\arabic{enumi})}
\setcounter{enumi}{5}
\tightlist
\item
  Which raw variable is explained the most by PC1 what about PC2
\end{enumerate}

\begin{Shaded}
\begin{Highlighting}[]
\KeywordTok{row.names}\NormalTok{(pca}\OperatorTok{$}\NormalTok{rotation)[}\KeywordTok{which.max}\NormalTok{(pca}\OperatorTok{$}\NormalTok{rotation[,}\DecValTok{1}\NormalTok{]}\OperatorTok{^}\DecValTok{2}\NormalTok{)]}
\end{Highlighting}
\end{Shaded}

\begin{verbatim}
## [1] "Petal.Length"
\end{verbatim}

Rather than reporting these values sometimes you will see a plot called
a variables factor map. It is a graphical representation of the
loadings. The package \texttt{FactoMineR} offers a nice way to produce
this plot.

\begin{enumerate}
\def\labelenumi{\arabic{enumi})}
\setcounter{enumi}{6}
\tightlist
\item
  Repeat your PCA using this package. In this package the pca is done
  with the function \texttt{PCA} and you can set \texttt{graph\ =\ TRUE}
  to autimatically produce the variables factor map.
\end{enumerate}

\begin{Shaded}
\begin{Highlighting}[]
\KeywordTok{library}\NormalTok{(FactoMineR)}
\NormalTok{pca3 <-}\StringTok{ }\KeywordTok{PCA}\NormalTok{(iris[,}\DecValTok{1}\OperatorTok{:}\DecValTok{4}\NormalTok{], }\DataTypeTok{graph =}\NormalTok{ T)}
\end{Highlighting}
\end{Shaded}

\includegraphics{pca-complete_files/figure-latex/unnamed-chunk-10-1.pdf}
\includegraphics{pca-complete_files/figure-latex/unnamed-chunk-10-2.pdf}

\hypertarget{input-data-and-assumptions}{%
\subsubsection{Input data and
assumptions}\label{input-data-and-assumptions}}

The third and fourth elements (\texttt{pca\$center} and
\texttt{pca\$scale}) of the list returned by \texttt{prcomp} are center
and scale these items describe the transformation that is performed on
the data before analysis.

PCA does make some assumptions about the input data the most important
of these is that measured variables either have linear or no
correlation. In practice, this is often untested. Instead, researchers
will often log transform all of their data before doing the PCA.

\begin{enumerate}
\def\labelenumi{\arabic{enumi})}
\setcounter{enumi}{7}
\tightlist
\item
  Try log transforming the iris data does this change the result?
\end{enumerate}

\begin{Shaded}
\begin{Highlighting}[]
\NormalTok{iris}\OperatorTok{$}\NormalTok{Sepal.Length <-}\StringTok{ }\KeywordTok{log}\NormalTok{(iris}\OperatorTok{$}\NormalTok{Sepal.Length)}
\NormalTok{iris}\OperatorTok{$}\NormalTok{Sepal.Width <-}\StringTok{ }\KeywordTok{log}\NormalTok{(iris}\OperatorTok{$}\NormalTok{Sepal.Width)}
\NormalTok{iris}\OperatorTok{$}\NormalTok{Petal.Length <-}\StringTok{ }\KeywordTok{log}\NormalTok{(iris}\OperatorTok{$}\NormalTok{Petal.Length)}
\NormalTok{iris}\OperatorTok{$}\NormalTok{Petal.Width <-}\StringTok{ }\KeywordTok{log}\NormalTok{(iris}\OperatorTok{$}\NormalTok{Petal.Width)}
\NormalTok{pca2 <-}\StringTok{ }\KeywordTok{prcomp}\NormalTok{(iris[, }\DecValTok{1}\OperatorTok{:}\DecValTok{4}\NormalTok{])}
\KeywordTok{plot}\NormalTok{(}\DataTypeTok{x =}\NormalTok{ pca2}\OperatorTok{$}\NormalTok{x[, }\DecValTok{1}\NormalTok{],}
     \DataTypeTok{y =}\NormalTok{ pca2}\OperatorTok{$}\NormalTok{x[, }\DecValTok{2}\NormalTok{],}
     \DataTypeTok{col =} \KeywordTok{rainbow}\NormalTok{(}\DecValTok{3}\NormalTok{)[iris}\OperatorTok{$}\NormalTok{Species],}
     \DataTypeTok{xlab =} \StringTok{"PC1"}\NormalTok{,}
     \DataTypeTok{ylab =} \StringTok{"PC2"}\NormalTok{,}
     \DataTypeTok{main =} \StringTok{"PCA of iris data"}\NormalTok{)}
\end{Highlighting}
\end{Shaded}

\includegraphics{pca-complete_files/figure-latex/unnamed-chunk-11-1.pdf}

\begin{enumerate}
\def\labelenumi{\arabic{enumi})}
\setcounter{enumi}{8}
\tightlist
\item
  PCA is quite sensitive to outliers. Lets alter one data measurement to
  illustrate this:
\end{enumerate}

\begin{Shaded}
\begin{Highlighting}[]
\KeywordTok{data}\NormalTok{(iris)}
\NormalTok{iris[}\DecValTok{150}\NormalTok{, }\DecValTok{1}\NormalTok{] <-}\StringTok{ }\DecValTok{200}
\end{Highlighting}
\end{Shaded}

\begin{enumerate}
\def\labelenumi{\arabic{enumi})}
\setcounter{enumi}{9}
\tightlist
\item
  Now perform the PCA again on this dataset and look at the result by
  plotting the data in the dimensions of the first two principal
  components.
\end{enumerate}

\begin{Shaded}
\begin{Highlighting}[]
\NormalTok{pca3 <-}\StringTok{ }\KeywordTok{prcomp}\NormalTok{(iris[, }\DecValTok{1}\OperatorTok{:}\DecValTok{4}\NormalTok{])}
\KeywordTok{plot}\NormalTok{(}\DataTypeTok{x =}\NormalTok{ pca3}\OperatorTok{$}\NormalTok{x[, }\DecValTok{1}\NormalTok{],}
     \DataTypeTok{y =}\NormalTok{ pca3}\OperatorTok{$}\NormalTok{x[, }\DecValTok{2}\NormalTok{],}
     \DataTypeTok{pch =} \DecValTok{16}\NormalTok{,}
     \DataTypeTok{col =} \KeywordTok{rainbow}\NormalTok{(}\DecValTok{3}\NormalTok{)[iris}\OperatorTok{$}\NormalTok{Species],}
     \DataTypeTok{xlab =} \StringTok{"PC1"}\NormalTok{,}
     \DataTypeTok{ylab =} \StringTok{"PC2"}\NormalTok{,}
     \DataTypeTok{main =} \StringTok{"PCA of iris data"}\NormalTok{)}
\end{Highlighting}
\end{Shaded}

\includegraphics{pca-complete_files/figure-latex/unnamed-chunk-13-1.pdf}

\hypertarget{clustering-data}{%
\subsubsection{Clustering data}\label{clustering-data}}

Often we would like to understand how well datapoints cluster in
principal component space. This is especially useful when we want to
determine whether a new data point falls within our existing categories.
To do this we will use a function \texttt{dataEllipse} from the
\texttt{car} package. Lets add an unknown specimen to our dataset and
try and determine which species we believe it belongs to.

\begin{enumerate}
\def\labelenumi{\arabic{enumi})}
\setcounter{enumi}{10}
\tightlist
\item
  follow this code to get a clean copy of your data and add a new
  datapoint from an unknown species.
\end{enumerate}

\begin{Shaded}
\begin{Highlighting}[]
\CommentTok{# lets clear our memory and start fresh}
\KeywordTok{rm}\NormalTok{(}\DataTypeTok{list=}\KeywordTok{ls}\NormalTok{())}

\KeywordTok{data}\NormalTok{(iris)}
\CommentTok{# first we need to convert the species names from}
\CommentTok{# factors to text}
\NormalTok{Species <-}\StringTok{ }\KeywordTok{as.character}\NormalTok{(iris}\OperatorTok{$}\NormalTok{Species)}

\CommentTok{# now we can add our new data}
\NormalTok{iris[}\DecValTok{151}\NormalTok{, }\DecValTok{1}\OperatorTok{:}\DecValTok{4}\NormalTok{] <-}\StringTok{ }\KeywordTok{c}\NormalTok{(}\DecValTok{7}\NormalTok{, }\FloatTok{3.1}\NormalTok{, }\FloatTok{4.5}\NormalTok{, }\FloatTok{1.3}\NormalTok{)}

\CommentTok{# and now we add the new set of species names as factors}
\NormalTok{iris}\OperatorTok{$}\NormalTok{Species <-}\StringTok{ }\KeywordTok{as.factor}\NormalTok{(}\KeywordTok{c}\NormalTok{(Species, }\StringTok{"unknown"}\NormalTok{))}
\end{Highlighting}
\end{Shaded}

\begin{enumerate}
\def\labelenumi{\arabic{enumi})}
\setcounter{enumi}{11}
\tightlist
\item
  Now repeat your PCA with this data. Plot the result of the PCA and add
  ellipses for each species. Below I illustrate the basic plot and one
  ellipse.
\end{enumerate}

\begin{Shaded}
\begin{Highlighting}[]
\KeywordTok{library}\NormalTok{(car)}
\end{Highlighting}
\end{Shaded}

\begin{verbatim}
## Loading required package: carData
\end{verbatim}

\begin{Shaded}
\begin{Highlighting}[]
\NormalTok{pca <-}\StringTok{ }\KeywordTok{prcomp}\NormalTok{(iris[,}\DecValTok{1}\OperatorTok{:}\DecValTok{4}\NormalTok{])}

\CommentTok{# examine first PCs}
\KeywordTok{plot}\NormalTok{(}\DataTypeTok{x =}\NormalTok{ pca}\OperatorTok{$}\NormalTok{x[, }\DecValTok{1}\NormalTok{],}
     \DataTypeTok{y =}\NormalTok{ pca}\OperatorTok{$}\NormalTok{x[, }\DecValTok{2}\NormalTok{],}
     \DataTypeTok{col =} \KeywordTok{c}\NormalTok{(}\StringTok{"red"}\NormalTok{,}\StringTok{"black"}\NormalTok{,}\StringTok{"green"}\NormalTok{, }\StringTok{"blue"}\NormalTok{)[iris}\OperatorTok{$}\NormalTok{Species],}
     \DataTypeTok{pch=}\DecValTok{16}\NormalTok{,}
     \DataTypeTok{cex=}\NormalTok{.}\DecValTok{5}\NormalTok{,}
     \DataTypeTok{xlab =} \StringTok{"PC1"}\NormalTok{,}
     \DataTypeTok{ylab =} \StringTok{"PC2"}\NormalTok{)}

\KeywordTok{dataEllipse}\NormalTok{(}\DataTypeTok{x =}\NormalTok{ pca}\OperatorTok{$}\NormalTok{x[}\DecValTok{1}\OperatorTok{:}\DecValTok{50}\NormalTok{, }\DecValTok{1}\NormalTok{],}
            \DataTypeTok{y =}\NormalTok{ pca}\OperatorTok{$}\NormalTok{x[}\DecValTok{1}\OperatorTok{:}\DecValTok{50}\NormalTok{, }\DecValTok{2}\NormalTok{],}
            \DataTypeTok{add =}\NormalTok{ T,}
            \DataTypeTok{plot.points =}\NormalTok{ F,}
            \DataTypeTok{levels =} \FloatTok{.95}\NormalTok{,}
            \DataTypeTok{center.pch =}\NormalTok{ F, }
            \DataTypeTok{col =} \StringTok{"red"}\NormalTok{,}
            \DataTypeTok{fill =}\NormalTok{ T,}
            \DataTypeTok{lwd=}\NormalTok{.}\DecValTok{5}\NormalTok{)}
\KeywordTok{dataEllipse}\NormalTok{(}\DataTypeTok{x =}\NormalTok{ pca}\OperatorTok{$}\NormalTok{x[}\DecValTok{51}\OperatorTok{:}\DecValTok{100}\NormalTok{, }\DecValTok{1}\NormalTok{],}
            \DataTypeTok{y =}\NormalTok{ pca}\OperatorTok{$}\NormalTok{x[}\DecValTok{51}\OperatorTok{:}\DecValTok{100}\NormalTok{, }\DecValTok{2}\NormalTok{],}
            \DataTypeTok{add =}\NormalTok{ T,}
            \DataTypeTok{plot.points =}\NormalTok{ F,}
            \DataTypeTok{levels =} \FloatTok{.95}\NormalTok{,}
            \DataTypeTok{center.pch =}\NormalTok{ F, }
            \DataTypeTok{col =} \StringTok{"green"}\NormalTok{,}
            \DataTypeTok{fill =}\NormalTok{ T,}
            \DataTypeTok{lwd=}\NormalTok{.}\DecValTok{5}\NormalTok{)}
\KeywordTok{dataEllipse}\NormalTok{(}\DataTypeTok{x =}\NormalTok{ pca}\OperatorTok{$}\NormalTok{x[}\DecValTok{101}\OperatorTok{:}\DecValTok{150}\NormalTok{, }\DecValTok{1}\NormalTok{],}
            \DataTypeTok{y =}\NormalTok{ pca}\OperatorTok{$}\NormalTok{x[}\DecValTok{101}\OperatorTok{:}\DecValTok{150}\NormalTok{, }\DecValTok{2}\NormalTok{],}
            \DataTypeTok{add =}\NormalTok{ T,}
            \DataTypeTok{plot.points =}\NormalTok{ F,}
            \DataTypeTok{levels =} \FloatTok{.95}\NormalTok{,}
            \DataTypeTok{center.pch =}\NormalTok{ F, }
            \DataTypeTok{col =} \StringTok{"blue"}\NormalTok{,}
            \DataTypeTok{fill =}\NormalTok{ T,}
            \DataTypeTok{lwd=}\NormalTok{.}\DecValTok{5}\NormalTok{)}
\end{Highlighting}
\end{Shaded}

\includegraphics{pca-complete_files/figure-latex/unnamed-chunk-15-1.pdf}

\end{document}
